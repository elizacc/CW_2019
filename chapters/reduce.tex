Построенный граф --- отображение пространства данных, которое мы хотим перевести в низкоразмерное пространство. Он упрощает работу тем, что мы теперь знаем, что нужно сделать, чтобы решить задачу: нам нужно как можно более точно перевести ребра с весами между объектами в новое пространство.

Как говорилось ранее, веса ребер можно воспринимать как вероятность существования данного ребра. Поэтому ребро $e$ является случайной величиной, имеющей распределение Бернулли: $e \sim B(w(e))$, где $w(e)$ --- вес ребра $e$. Получается, что множество ребер построенного графа --- множество $E$ из случайных величин Бернулли \cite{umap}.

Функция веса зависит от аргументов, рассчитанных для признаков конкретного пространства. По этой причине, функция веса в низкоразмерном пространстве будет отличаться от ее аналога в высокоразмерном пространстве.

Тогда для более корректного переноса данных мы можем подобрать для множества~$E_h$ похожее на него множество~$E_l$ с функцией~$w_l(e)$, соответствующие низкоразмерному пространству. При этом $E_h$ и $E_l$ состоят из одинаковых элементов (ребер). 

Здесь нам пригодится суммарная дивергенция Кульбака-Лейблера $C_{E_h}(E_l)$. Мы можем решать задачу минимизации, с целью приблизить новую функцию весов к старой \cite{umap}:
\begin{equation}
	C_{E_h}(E_l) = \sum_{e \in E} w_h(e) \log \frac{w_h(e)}{w_l(e)} + (1 - w_h(e)) \log \left(\frac{1 - w_h(e)}{1 - w_l(e)}\right) \rightarrow \min_{w_l}
\end{equation}

Приведем задачу к минимизации кросс-энтропии:
\begin{equation}
	-\sum_{e \in E} \left(w_h(e)\log w_l(e) + (1 - w_h(e))\log (1 - w_l(e))\right) \rightarrow \min_{w_l}
\end{equation}

Рассмотрим отдельно производные слагаемых внутри суммы (для одного ребра):
\begin{itemize}
	\item $\displaystyle (-w_h(e)\log w_l(e))' = - \frac{w_h(e)}{w_l(e)} < 0$
	
	Значит, первое слагаемое убывает по $w_l(e) \Rightarrow$ чем больше $w_l(e)$, тем меньше становится первое слагаемое $\Rightarrow$ при увеличении $w_l(e)$ первое слагаемое уменьшает кросс-энтропию, увеличивая сходство множеств.
	
	Авторы назвали данное явление сближающей силой, которая сдвигает точки ребра в новом пространстве. Увеличивая вес, мы уменьшаем расстояние между точками, делая их близкими соседями.
	
	\item $\displaystyle (-(1 - w_h(e))\log (1 - w_l(e)))' = \frac{1 - w_h(e)}{1 - w_l(e)} > 0$
	
	Значит, второе слагаемое возрастает по $w_l(e) \Rightarrow$ чем меньше $w_l(e)$, тем меньше становится второе слагаемое $\Rightarrow$ при уменьшении $w_l(e)$ второе слагаемое уменьшает кросс-энтропию, увеличивая сходство множеств.
	
	Данное явление авторы назвали отталкивающей силой, которая раздвигает точки ребра. Уменьшая вес, мы увеличиваем расстояние между точками. Они отодвигаются друг от друга, но остаются соседями, так как были ими в исходном пространстве.	
\end{itemize}

Теперь мы можем создать некоторое отображение точек в низкоразмерном пространстве и применить к ним данные силы. Сближая и отталкивая точки, UMAP постепенно будет приводить их к нужному виду \cite{umap}. И работа алгоритма может быть остановлена тогда, когда не будет наблюдаться значительных изменений в расположении точек (изменение в расположении будет не больше $\epsilon$). На этом работа UMAP заканчивается --- у нас получилось снизить размерность данных!