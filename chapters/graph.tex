На вход в алгоритм поступает набор данных: $X = \{x_1, ..., x_n\}$. UMAP выполняет построение ориентированного взвешенного графа: вершинами являются объекты, ребрами --- связи между ними. Алгоритм рассчитывает расстояния между всеми объектами по заданной метрике $d: X \times X \rightarrow \mathbb{R_+}$, а затем определяет для каждого список из $k$ ближайших соседей: $\{t_1, ..., t_k\}$. Каждая вершина соединяется ребром с каждой из своего списка. Таким образом, не все вершины оказываются связаны друг с другом, но каждая имеет соседей. UMAP использует градиентный спуск для поиска ближайших соседей \cite{mcinnes}.

Помимо списка из ближайших вершин для каждого объекта рассчитывается расстояние до ближайшего соседа, не совпадающего с ним: 
\begin{equation}
	\rho_i = \min_{j \in [1,k]} d(x_i, t_j)
\end{equation}

И величина $\sigma_i > 0$, которая задается уравнением: 
\begin{equation}
	\sum_{j=1}^{k} \exp\left(-\frac{d(x_i, t_j) - \rho_i}{\sigma_i}\right) = \log_2 k
\end{equation}

С помощью $\sigma_i$ осуществляется нормировка расстояний между объектами для расчёта весов. Если $\sigma_i \in (0,1)$, то нормированные расстояния между объектами увеличиваются. При $\sigma_i > 1$ --- уменьшаются.

Поэтому $\sigma_i$ можно интерпретировать, как величину, показывающую, насколько тесно располагаются точки соседства. Если они лежат близко ($\sigma_i \in (0,1)$), то нормировка <<раздвигает>> их, если далеко ($\sigma_i > 1$), то <<сближает>> --- необходимо помнить, что нормировка нужна для расчета весов и никак не влияет на исходное пространство.

Вес ребра определяется следующим образом:
\begin{equation}
	w(x_i \rightarrow t_j) = \exp\left(-\frac{d(x_i, t_j) - \rho_i}{\sigma_i}\right)
\end{equation}

То есть $\sigma_i$ задается таким образом, что сумма весов для каждого объекта нормируется к одному и тому же числу $\log_2 k$.

Разберемся, почему вес задается именно такой формулой:

\begin{enumerate}
	\item $(\displaystyle d(x_i, t_j) - \rho_i)$ --- относительное расстояние до $j$-ого соседа (по отношению к расстоянию до ближайшего соседа).
	
	Происходит смещение центра соседства в точку, соответствующую ближайшему соседу. Это нормирует вес ребра до ближайшего соседа к 1 (по конечной формуле), что является максимально возможным значением веса.
	
	\item $\displaystyle\frac{d(x_i, t_j) - \rho_i}{\sigma_i}$ --- относительное расстояние до $j$-ого соседа, нормированное на $\sigma_i$.
	
	Для всех объектов расстояния до соседей стандартизируются, а также сумма весов приравнивается к одному числу, что позволяет сравнивать объекты и применять к ним единую схему снижения размерности.
	\item $\displaystyle \exp\left(-\frac{d(x_i, t_j) - \rho_i}{\sigma_i}\right)$ --- вес ребра от $i$-ого объекта к его $j$-ому соседу.
	
	Экспоненциальная функция вновь нормирует веса, располагая их в полуинтервале $(0, 1]$, а также выстраивает обратное отношение между весом и отдаленностью соседа. Теперь, чем ближе сосед, тем больше его вес --- расстояние и вес обратно пропорциональны.
	
	Именно поэтому $\sigma_i < 0$ не подходит, так как в этом случае при росте расстояния между объектами $d(x_i, t_j)$ вес объекта тоже будет увеличиваться --- нам нужно обратное.
\end{enumerate}

Однако получилось так, что ребрам, соединяющим одни и те же точки, могут соответствовать разные веса, поскольку формула веса несимметрична относительно объектов. Если объекты $a$ и $b$ соединены ребрами в обе стороны, то $\displaystyle w(a \rightarrow b) = \exp\left(-\frac{d(a, b) - \rho_a}{\sigma_a}\right)$, а $\displaystyle w(b \rightarrow a) = \exp\left(-\frac{d(b, a) - \rho_b}{\sigma_b}\right)$.

Перевод обоих ребер для всех объектов затрачивает гораздо больше ресурсов, чем если бы мы переводили лишь одно.

Чтобы оставить только одно, нам необходимо объединить веса. Вес ребра можно интерпретировать как вероятность существования данного ребра: $w_b(a, b)$ --- вероятность того, что в окрестности точки $a$ появится объект $b$ (и между ними возникнет ребро) \cite{hansen}. Поэтому новый вес можно представить, как вероятность того, что хотя бы одно ребро существует, считая, что существование ребра $a \rightarrow b$ и существование ребра $b \rightarrow a$ --- независимые события:
\begin{equation}
	w(a,b) = w(a \rightarrow b) + w(b \rightarrow a) - w(a \rightarrow b) \cdot w(b \rightarrow a)
\end{equation}

Итак, после преобразования весов мы получили взвешенный неориентированный граф. Перейдем к части снижения размерности.