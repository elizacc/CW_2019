Снижение размерности данных широко используется в анализе данных. Существующие алгоритмы применяют самые разные подходы. В частности, можно разделить алгоритмы следующим образом \cite{reducebasics}:
\begin{enumerate}
	\item Отбор признаков (feature selection) --- алгоритмы выбирают из имеющихся в выборке признаков наиболее значимые, наиболее информативные
	\item Извлечение признаков (feature extraction) --- алгоритмы создают новые признаки на основе тех, что были в выборке. Их задача: сохранить как можно больше информации в новых признаках
\end{enumerate}

Алгоритм UMAP, который мы будем рассматривать, относится ко второму типу: он создает новые признаки на основе исходных с помощью нелинейных преобразований.

Помимо этой классификации существует множество других. Каждый отдельный алгоритм по-своему уникален, но все они работают над одной целью --- снизить размерность данных.

Но зачем нужно снижение размерности? Есть много задач, которые оно решает \cite{reducebasics}:
\begin{itemize}
	\item \textbf{Выделение наиболее важных признаков}
	
	В результате снижения размерности алгоритмы отбора признаков оставляют только те признаки, которые несут себе больше всего информации об объектах. Это может пригодиться для дальнейшего анализа данных.
	
	\item \textbf{Проклятие размерности}
	
	Количество признаков в выборке определяет размерность пространства, в котором находятся данные. Чем больше признаков, тем больше информации про объекты --- тем больше времени нужно на её обработку.
	
	Например, у нас есть 20 объектов, один признак, который принимает два значения (1 и 0) --- у нас есть всего два типа объектов: 1 и 0. При появлении второго признака с двумя значениями (A и B) вариантов становится в два раза больше: 1А, 1В, 2А, 2В. При появлении других признаков объем данных растет экспоненциально.
	
	Снижение размерности позволяет избежать проклятия размерности.
	
	\item \textbf{Визуализация данных}
	
	Данные высокой размерности невозможно визуализировать, из-за чего их сложно интерпретировать. Снижение размерности до $n=1,2,3$ позволяет изобразить выборку на графике, чтобы посмотреть, что она из себя представляет.
\end{itemize}

И это не все бонусы снижения размерности. Далее мы подробнее рассмотрим один из алгоритмов (UMAP), но в начале немного теории.