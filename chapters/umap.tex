\graphicspath{{main/}}
\textbf{UMAP (Uniform Manifold Approximation and Projection)} --- алгоритм снижения размерности, базирующийся на использовании кросс-энтропии при аппроксимации низкоразмерного пространства. Он позволяет переместить объекты из высокоразмерного пространства в низкоразмерное.
\begin{center}
\begin{tabular}{|c|p{0.5cm}|p{0.5cm}|p{0.5cm}|p{0.5cm}|p{0.5cm}|p{0.6cm}|c|p{0.5cm}|p{0.5cm}|p{0.5cm}|}
	\multicolumn{11}{c}{\textbf{Схема работы UMAP}}\\
	\multicolumn{11}{c}{ }\\
	\cline{1-6}
	\cline{8-11}
	& \multicolumn{5}{c|}{$D$ признаков} & \multirow{7}{2cm}{$\longrightarrow$} & & \multicolumn{3}{c|}{$d$ признаков}\\
	\cline{1-6}
	\cline{8-11}
	\multirow{6}{*}{$k$ объектов} & & & & & & & \multirow{6}{*}{$k$ объектов} & & &\\
	\cline{2-6}
	\cline{9-11}
	 & & & & & & & & & &\\
	\cline{2-6}
	\cline{9-11}
	 & & & & & & & & & &\\
	\cline{2-6}
	\cline{9-11}
	 & & & & & & & & & &\\
	\cline{2-6}
	\cline{9-11}
	 & & & & & & & & & &\\
	\cline{2-6}
	\cline{9-11}
	& & & & & & & & & &\\
	\cline{1-6}
	\cline{8-11}
	\multicolumn{11}{c}{ }\\
	\multicolumn{11}{c}{Здесь $d \ll D$}\\
\end{tabular}
\end{center}


Помимо UMAP, существуют другие алгоритмы, позволяющие переводить данные в низкоразмерное пространство. Например, t-SNE, PCA (метод главных компонент) и другие. Но алгоритм UMAP имеет ряд преимуществ \cite{mcinnes}:

\begin{itemize}
	\item \textbf{Использование нелинейных функций от признаков}
	
	Алгоритмы снижения размерности переводят все имеющиеся у объектов признаки в $d$ признаков: $f: X^D \rightarrow Y^d$. При этом каждый алгоритм делает это по-своему. В отличие от PCA, UMAP использует не только линейные функции $f$ для перевода, что позволяет более точно сохранить информацию об исходных признаках.
	
	\item \textbf{Время работы}
	
	Алгоритмы снижения размерности требуют больших затрат времени. Однако UMAP, в отличие от t-SNE, способен быстрее справляться с задачей. Так, UMAP обработал набор данных из 70000 изображений по 784 пикселя (признака) за 2,5 минуты. Работая на том же компьютере, t-SNE на данную задачу потребовалось 45 минут.
	
	\item \textbf{Высокая размерность данных}
	
	По словам авторов, ограничения на начальную размерность пространства у UMAP отсутствуют, что дает ему большое преимущество перед t-SNE\cite{mcinnes}.
\end{itemize}

