Данный алгоритм может регулироваться за счет установления гиперпараметров \cite{umap}. К основным относятся:
\begin{itemize}
	\item $\verb|n_neighbors|$ --- количество ближайших соседей. Данный гиперпараметр определяет количество объектов, которые находятся ближе всего к анализируемому (близость определяется расстоянием, посчитанным по выбранной метрике). Используется для расчета локальных связей и переноса их в новое пространство. Для краткости будем обозначать количество соседей буквой k, по аналогии с алгоритмом kNN.\\[1mm]
	Маленькое значение k приводит к переобучению --- алгоритм уделяет больше внимания локальной структуре, пренебрегая глобальной. То есть наличие взаимосвязи между объектами, находящимися не рядом, будет потеряно в новом пространстве.\\[1mm]	
	Большое значение k приводит к недообучению. UMAP рассматривает большую окрестность рядом с анализируемым объектом, и меньше внимания уделяет локальным связям. То есть, алгоритм будет пренебрегать взаимосвязями между объектами, находящимися рядом, оставляя глобальные закономерности.\\[1mm]	
	По умолчанию, \verb|n_neighbors=15|. Это универсальное значение для выборок с большим количеством объектов (значительно больше 15 объектов), которое как сохраняет локальную структуру, так и переносит глобальные взаимосвязи. Но для каждой отдельной выборки параметр стоит подбирать индивидуально.
	
	\item \verb|metric| --- метрика, используемая для расчета расстояний между объектами. Выбирается в зависимости от типа анализируемых данных.\\[1mm]	
	По умолчанию, \verb|metric = "euclidean"|. Самая распространенная метрика, которая может применяться к любой выборке с вещественными признаками. Для других типов признаков, или по собственному желанию, метрика может меняться. В том числе может быть использована метрика, созданная самим пользователем.
	
	\item \verb|n_components| --- размерность нового пространства. Позволяет задать количество признаков в пространстве с меньшей размерностью. Если после обработки информации необходимо визуализировать её, то можно выбрать $d=2$, $d=3$ или даже $d=1$. Для дальнейшей обработки данных подходит и $d>3$.\\[1mm]	
	По умолчанию, \verb|n_components=2|. Удобно для визуализации на плоскости.
	
	\item \verb|min_dist| --- расстояние между точками в низкоразмерном пространстве, лежит в отрезке $[0,1]$. Используется, чтобы изменять изображение, визуализирующее результат работы алгоритма.\\[1mm]	
	Маленькое значение \verb|min_dist| сближает точки низкоразмерного пространства. В том числе точки из одного кластера, из-за чего мы можем увидеть разбиение данных на кластеры (если такое разбиение возможно).\\[1mm]	
	Большое значение \verb|min_dist| раздвигает точки и позволяет посмотреть на структуру данных в целом, увидеть какие области находятся рядом, какие далеко.\\[1mm]	
	По умолчанию, \verb|min_dist=0.1|. При таком значении, можно как рассмотреть кластеры при наличии, так и увидеть глобальную структуру.
\end{itemize}

Почитать про остальные гиперпараметры можно \href{https://umap-learn.readthedocs.io/en/latest/api.html}{здесь}. В данной работе мы будем рассматривать только основные.