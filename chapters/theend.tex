Принцип работы алгоритма основывается на построении взвешенного графа в многомерном пространстве, а затем создании его аналога в низкоразмерном. Задача минимизации кросс-энтропии помогает приблизить новый граф к исходному.

Реализация алгоритма, основанная на методе ближайших соседей, позволяет не учитывать взаимосвязи между непохожими объектами и, соответственно, уменьшить время обработки данных, что дает UMAP значительное преимущество в скорости.

В то же время из-за того, что данный метод отбрасывает ненужные связи, при малом количестве соседей даже не похожие объекты могут оказаться ближе, чем похожие. Поэтому необходимо грамотно подбирать основной гиперпараметр алгоритма --- \verb|n_neighbors|. По этой же причине при анализе результатов стоит помнить о том, что алгоритм меряет схожесть, а не отличие --- можно делать выводы только о схожести объектов. Также стоит помнить о разнице между корреляцией и схожести в понятиях UMAP. 

Но несмотря на перечисленные ограничения, использование UMAP может быть очень полезно при анализе любого типа данных. Его применение позволяет снизить размерность исходной выборки, посмотреть на их внутреннее устройство и отследить некоторые закономерности.