В алгоритме UMAP используется дивергенция Кульбака-Лейблера для случайной величины Бернулли $X \sim B(p(x))$ \cite{umap}. Здесь $p(x)$ --- некоторая функция, определяющая вероятность того, что $X=1$. Запишем формулу $D_{KL}$ для $X$:
\begin{multline}
	D_{KL}(P\, ||\, \tilde P)= - p(x)\log \tilde p(x) - (1 - p(x))\log (1 - \tilde p(x)) + p(x)\log p(x) + (1 - p(x))\log (1 - p(x)) = \\
	= p(x)\log \frac{p(x)}{\tilde p(x)} + (1 - p(x))\log \frac{1 - p(x)}{1 - \tilde p(x)}
\end{multline}

Однако алгоритм рассчитывает не просто разницу между двумя распределениями для одной случайной величины, а сумму таких разниц для $n$ случайных величин (для двух множеств из $n$ случайных величин: $S$ и $\tilde S$):
\begin{equation}
	C_S(\tilde S) = \sum_{i=1}^n \left(p(x_i)\log \frac{p(x_i)}{\tilde p(x_i)} + (1 - p(x_i))\log \frac{1 - p(x_i)}{1 - \tilde p(x_i)}\right)
\end{equation}

Данная величина\footnote{Авторы алгоритма в своей статье называют множества из случайных величин нечеткими множествами. А $C_S(\tilde S)$ --- кросс-энтропией нечетких множеств (fuzzy set cross entropy) \cite{mcinnes}.} показывает степень отдаленности множества $S$ от множества $\tilde S$. При этом каждому множеству соответствует своя функция: $p(x)$ и $\tilde p(x)$. 

Как и дивергенция Кульбака-Лейблера, суммарная дивергенция является несимметричной функцией. Однако это нисколько не мешает использовать её.

Минимизация $C_S(\tilde S)$ по $\tilde p(x)$ позволяет найти множество $\tilde S$, которое наиболее похоже на множество $S$. В таком случае известным является нечеткое множество $S$, с которым мы сравниваем, а искомым --- то, которое сравниваем $\tilde S$.

Задача минимизации заключается в поиске оптимального $\tilde p(x)$. Если вернуться к исходной записи $D_{KL} = H_P(\tilde P) - H(P)$, то видно, что энтропия не зависит от $\tilde p(x)$, соответственно, является константой при минимизации. Тогда задача преобразуется в оптимизацию лишь суммы кросс-энтропий:
\begin{equation}
	-\sum_{i=1}^n \left(p(x_i)\log \tilde p(x_i) + (1 - p(x_i))\log (1 - \tilde p(x_i))\right) \rightarrow \min_{\tilde p}
\end{equation}

Поэтому UMAP считается методом, основанным на кросс-энтропии.